%% -*- coding: utf-8 -*-
\documentclass{beamer}

%% -*- coding: utf-8 -*-
\usetheme{Boadilla} % default
\useoutertheme{infolines}
\setbeamertemplate{navigation symbols}{} 

\usepackage{etex}
\usepackage{alltt}
\usepackage{pifont}
\usepackage{color}
\usepackage[utf8]{inputenc}
\usepackage{german}
\usepackage{listings}
\usepackage{hyperref}
\hypersetup{colorlinks=true}
\usepackage[final]{pdfpages}
\usepackage{url}
\usepackage{arydshln} % dashed lines

\newcommand\cmark{\ding{51}}
\newcommand\xmark{\ding{55}}

\newcommand{\nat}{\mathbf{N}}

\usepackage[all]{xy}

%% new arrow tip for xy
\newdir{|>}{!/4.5pt/@{|}*:(1,-.2)@^{>}*:(1,+.2)@_{>}}

\newcommand\cid[1]{\textup{\textbf{#1}}} % class names
\newcommand\kw[1]{\textup{\textbf{#1}}}  % key words
\newcommand\tid[1]{\textup{\textsf{#1}}} % type names
\newcommand\vid[1]{\textup{\texttt{#1}}} % value names
\newcommand\Mid[1]{\textup{\texttt{#1}}} % method names

\newcommand\TODO[1][]{{\color{red}{\textbf{TODO: #1}}}}

\newcommand\String[1]{\texttt{\dq{}#1\dq{}}}

\newcommand\ClassHead[1]{%
  \ensuremath{\begin{array}{|l|}
      \hline
      \cid{#1}
      \\\hline
    \end{array}}}
\newcommand\AbstractClass[2]{%
  \ensuremath{\begin{array}{|l|}
      \hline
      \cid{\textit{#1}}
      \\\hline
      #2
      \hline
    \end{array}}}
\newcommand\Class[2]{%
  \ensuremath{\begin{array}{|l|}
      \hline
      \cid{#1}
      \\\hline
      #2
      \hline
    \end{array}}}
\newcommand\Attribute[3][black]{\textcolor{#1}{\Param{#2}{#3}}\\}
\newcommand\Methods{\hline}
\newcommand\MethodSig[3]{\Mid{#2} (#3): \,\tid{#1}\\}
\newcommand\CtorSig[2]{\Mid{#1} (#2)\\}
\newcommand\AbstractMethodSig[3]{\Mid{\textit{#2}} (#3): \,\tid{#1}\\}
\newcommand\Param[2]{\vid{#2}:~\tid{#1}}

\lstset{%
  frame=single,
  xleftmargin=2pt,
  stepnumber=1,
  numbers=left,
  numbersep=5pt,
  numberstyle=\ttfamily\tiny\color[gray]{0.3},
  belowcaptionskip=\bigskipamount,
  captionpos=b,
  escapeinside={*'}{'*},
  language=java,
  tabsize=2,
  emphstyle={\bf},
  commentstyle=\mdseries\it,
  stringstyle=\mdseries\rmfamily,
  showspaces=false,
  showtabs=false,
  keywordstyle=\bfseries,
  columns=fullflexible,
  basicstyle=\footnotesize\CodeFont,
  showstringspaces=false,
  morecomment=[l]\%,
  rangeprefix=////,
  includerangemarker=false,
}

\newcommand\CodeFont{\sffamily}

\definecolor{lightred}{rgb}{0.8,0,0}
\definecolor{darkgreen}{rgb}{0,0.5,0}
\definecolor{darkblue}{rgb}{0,0,0.5}

\newcommand\highlight[1]{\textcolor{blue}{\emph{#1}}}
\newcommand\GenClass[2]{\cid{#1}\texttt{<}\cid{#2}\texttt{>}}

\newcommand\Colored[3]{\alt<#1>{\textcolor{#2}{#3}}{#3}}

\newcommand\nt[1]{\ensuremath{\langle#1\rangle}}

%%% Local Variables: 
%%% mode: latex
%%% TeX-master: nil
%%% End: 

%%% frontmatter
%% -*- coding: utf-8 -*-

\title{Functional Programming}
\subtitle{Introduction}

\author[Peter Thiemann]{Prof. Dr. Peter Thiemann}
\institute[Univ. Freiburg]{Albert-Ludwigs-Universität Freiburg, Germany}
\date{WS 2022/23}


\subtitle{IO}
\usepackage{tikz}


\begin{document}

\begin{frame}
  \titlepage
\end{frame}
%----------------------------------------------------------------------
\begin{frame}[fragile]
  \frametitle{Referential transparency and substitutivity}
  \begin{block}<+->{Remember the first class?}
    \begin{itemize}
    \item Every variable and expression has just one value\\
      \textbf{referential transparency}
    \item Every variable can be replaced by its definition\\
      \textbf{substitutivity}
    \end{itemize}
  \end{block}
  \begin{block}<+->{Enables reasoning}
\begin{verbatim}
-- sequence of function calls does not matter
f () + g () == g () + f ()
-- number of function calls does not matter
f () + f ( ) == 2 * f ()
\end{verbatim}
  \end{block}
\end{frame}
%----------------------------------------------------------------------
\begin{frame}[fragile]
  \frametitle{How does IO fit in?}
  \begin{alertblock}<+->{Bad example}
    Suppose we had 
\begin{verbatim}
input :: () -> Integer
\end{verbatim}
    \begin{itemize}
    \item<+-> Consider
\begin{verbatim}
let x = input () in
x + x
\end{verbatim}
    \item<+-> Expect to read one input and use it twice
    \item<+-> By substitutivity, this expression must behave like
\begin{verbatim}
input () + input ()
\end{verbatim}
      which reads two inputs!
    \item<+-> VERY WRONG!!!
  \end{itemize}
  \end{alertblock}
\end{frame}
%----------------------------------------------------------------------
\begin{frame}[fragile]
  \frametitle{The dilemma}
  \begin{block}<+->{Haskell is a pure language, but IO is a side effect}
  \end{block}
  \begin{block}<+->{A contradiction?}
  \end{block}
  \begin{block}<+->{No!}
    \begin{itemize}
    \item Instead of performing IO operations directly, there is an
      abstract type of \textbf{IO instructions}, which get executed
      lazily by the operating system
    \item Some instructions (e.g., read from a file) return values, so the abstract IO type is parameterized over their type
    \item Keep in mind: instructions are just values like any other
  \end{itemize}
  \end{block}
\end{frame}
%----------------------------------------------------------------------
\begin{frame}[fragile]
  \frametitle{Haskell IO}

\begin{block}<+->{The main function}
  Top-level result of a program is an IO ``instruction''.
\begin{verbatim}
main :: IO ()
main = undefined
\end{verbatim}
  \begin{itemize}
  \item an instruction describes the \textbf{effect} of the program
  \item effect $=$ IO action, imperative state change, \dots
  \end{itemize}
\end{block}
% \begin{block}<+->{An instruction that returns a result}
% \begin{verbatim}
% -- defined in the Prelude
% readFile :: FileName -> IO String
% \end{verbatim}
% \end{block}
\end{frame}
%----------------------------------------------------------------------
\begin{frame}[fragile]
  \frametitle{Kinds of instructions}
  \begin{block}<+->{Primitive instructions}
\begin{verbatim}
-- defined in the Prelude
putChar   :: Char -> IO ()
getChar   :: IO Char
writeFile :: FileName -> String -> IO ()
readFile  :: FileName -> IO String
\end{verbatim}
and many more
  \end{block}
  \begin{block}<+->{No op instruction}
\begin{verbatim}
return :: a -> IO a
\end{verbatim}
    The IO instruction \texttt{return 42} performs no IO, but yields the value 42.
  \end{block}
\end{frame}
%----------------------------------------------------------------------
\begin{frame}[fragile]
  \frametitle{Combining two instructions}
  \begin{block}<+->{The bind operator \texttt{>>=}}
    Intuition:  next instruction may depend on the output of the previous one
\begin{verbatim}
(>>=) :: IO a -> (a -> IO b) -> IO b
\end{verbatim}
    The instruction \texttt{m >>= f}
    \begin{itemize}
    \item executes \texttt{m :: IO a} first
    \item gets its result \texttt{x :: a}
    \item applies \texttt{f :: a -> IO b} to the result
    \item to obtain an instruction \texttt{f x :: IO b} that returns a \texttt{b}
    \item and executes this instruction to return a \texttt{b}
    \end{itemize}
  \end{block}
  \begin{block}<+->{Example}
\begin{verbatim}
readFiles f1 f2 =
   readFile f1 >>= \xs1 -> readFile f2 
\end{verbatim}
  \end{block}
\end{frame}
%----------------------------------------------------------------------
\begin{frame}[fragile]
  \frametitle{More convenient: \texttt{do} notation}
\begin{verbatim}
copyFile source target =
  undefined

doTwice io =
  undefined

doNot io =
  undefined
\end{verbatim}
\end{frame}
%----------------------------------------------------------------------
\begin{frame}
  \frametitle{Translating \texttt{do} notation into \texttt{>>=} operations}
  \begin{itemize}
  \item \texttt{do \emph{action}}
    \\ $\longrightarrow$
    \\ \texttt{\emph{action}}
  \item \texttt{do \{ \emph{x} <- \emph{action1}; \emph{instructions} \}}
    \\ $\longrightarrow$
    \\ \texttt{\emph{action1} >>= \textbackslash \emph{x} -> \texttt{do \{ \emph{instructions} \}}}
  \item \texttt{do \{ let \emph{binding}; \emph{instructions} \}}
    \\ $\longrightarrow$
    \\ \texttt{let \emph{binding} in do \{  \emph{instructions} \}}
  \end{itemize}
\end{frame}
%----------------------------------------------------------------------
\begin{frame}[fragile]
  \frametitle{Instructions vs functions}
  \begin{block}<+->{Functions}
    behave the same each time they called
  \end{block}
  \begin{block}<+->{Instructions}
    may be interpreted differently each time 
    they are executed, depending on context    
  \end{block}
\end{frame}
%----------------------------------------------------------------------
\begin{frame}
  \frametitle{Underlying concept: \textbf{Monad}}

  \begin{block}<+->{What's a monad?}
    \begin{itemize}
    \item abstract datatype for instructions that produce values
    \item built-in combination \texttt{>>=}
    \item abstracts over different interpretations (computations)
    \end{itemize}
  \end{block}
  \begin{alertblock}<+->{IO is a special case of a monad}
    \begin{itemize}
    \item one very useful application for monad
    \item built into Haskell
    \item but there's more to the concept
    \item many more instances to come!
    \end{itemize}
  \end{alertblock}
\end{frame}
%----------------------------------------------------------------------
\begin{frame}[fragile]
  \frametitle{Hands-on task}
  Define a function
\begin{verbatim}
sortFile :: FilePath -> FilePath -> IO ()

-- sortFile inFile outFile 
-- reads inFile, sorts its lines, and writes the result to outFile

-- recall
-- sort :: Ord a => [a] -> [a]
-- lines :: String -> [String]
-- unlines :: [String] -> String
\end{verbatim}
\end{frame}
%----------------------------------------------------------------------
\begin{frame}[fragile]
  \frametitle{Utilities}
\begin{verbatim}
sequence :: [IO a] -> IO [a]
sequence_ :: [IO a] -> IO ()
\end{verbatim}
\end{frame}
%----------------------------------------------------------------------
\begin{frame}[fragile]
  \frametitle{Another hands-on task}
  Define a function
\begin{verbatim}
printTable :: [String] -> IO ()

{-
printTable ["New York", "Rio", "Tokio"]
outputs
1: New York
2: Rio
3: Tokio
-} 
\end{verbatim}
\end{frame}
%----------------------------------------------------------------------
%----------------------------------------------------------------------
\begin{frame}
  \frametitle{Wrapup}
  \begin{alertblock}{First meeting with monads}
  \begin{itemize}
  \item abstract data type of instructions returning results
  \item next instruction can depend on previous result
  \item instructions are just values
  \item basis for Haskell's standard IO
  \end{itemize}
  \end{alertblock}
\end{frame}

% \begin{frame}
%   \frametitle{Break Time --- Questions?}
%   \begin{center}
%     \tikz{\node[scale=15] at (0,0){?};}
%   \end{center}
% \end{frame}


\end{document}

%%% Local Variables: 
%%% mode: latex
%%% TeX-master: t
%%% End: 
