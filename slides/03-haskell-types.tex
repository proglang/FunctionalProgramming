%% -*- coding: utf-8 -*-
\documentclass{beamer}

%% -*- coding: utf-8 -*-
\usetheme{Boadilla} % default
\useoutertheme{infolines}
\setbeamertemplate{navigation symbols}{} 

\usepackage{etex}
\usepackage{alltt}
\usepackage{pifont}
\usepackage{color}
\usepackage[utf8]{inputenc}
\usepackage{german}
\usepackage{listings}
\usepackage{hyperref}
\hypersetup{colorlinks=true}
\usepackage[final]{pdfpages}
\usepackage{url}
\usepackage{arydshln} % dashed lines

\newcommand\cmark{\ding{51}}
\newcommand\xmark{\ding{55}}

\newcommand{\nat}{\mathbf{N}}

\usepackage[all]{xy}

%% new arrow tip for xy
\newdir{|>}{!/4.5pt/@{|}*:(1,-.2)@^{>}*:(1,+.2)@_{>}}

\newcommand\cid[1]{\textup{\textbf{#1}}} % class names
\newcommand\kw[1]{\textup{\textbf{#1}}}  % key words
\newcommand\tid[1]{\textup{\textsf{#1}}} % type names
\newcommand\vid[1]{\textup{\texttt{#1}}} % value names
\newcommand\Mid[1]{\textup{\texttt{#1}}} % method names

\newcommand\TODO[1][]{{\color{red}{\textbf{TODO: #1}}}}

\newcommand\String[1]{\texttt{\dq{}#1\dq{}}}

\newcommand\ClassHead[1]{%
  \ensuremath{\begin{array}{|l|}
      \hline
      \cid{#1}
      \\\hline
    \end{array}}}
\newcommand\AbstractClass[2]{%
  \ensuremath{\begin{array}{|l|}
      \hline
      \cid{\textit{#1}}
      \\\hline
      #2
      \hline
    \end{array}}}
\newcommand\Class[2]{%
  \ensuremath{\begin{array}{|l|}
      \hline
      \cid{#1}
      \\\hline
      #2
      \hline
    \end{array}}}
\newcommand\Attribute[3][black]{\textcolor{#1}{\Param{#2}{#3}}\\}
\newcommand\Methods{\hline}
\newcommand\MethodSig[3]{\Mid{#2} (#3): \,\tid{#1}\\}
\newcommand\CtorSig[2]{\Mid{#1} (#2)\\}
\newcommand\AbstractMethodSig[3]{\Mid{\textit{#2}} (#3): \,\tid{#1}\\}
\newcommand\Param[2]{\vid{#2}:~\tid{#1}}

\lstset{%
  frame=single,
  xleftmargin=2pt,
  stepnumber=1,
  numbers=left,
  numbersep=5pt,
  numberstyle=\ttfamily\tiny\color[gray]{0.3},
  belowcaptionskip=\bigskipamount,
  captionpos=b,
  escapeinside={*'}{'*},
  language=java,
  tabsize=2,
  emphstyle={\bf},
  commentstyle=\mdseries\it,
  stringstyle=\mdseries\rmfamily,
  showspaces=false,
  showtabs=false,
  keywordstyle=\bfseries,
  columns=fullflexible,
  basicstyle=\footnotesize\CodeFont,
  showstringspaces=false,
  morecomment=[l]\%,
  rangeprefix=////,
  includerangemarker=false,
}

\newcommand\CodeFont{\sffamily}

\definecolor{lightred}{rgb}{0.8,0,0}
\definecolor{darkgreen}{rgb}{0,0.5,0}
\definecolor{darkblue}{rgb}{0,0,0.5}

\newcommand\highlight[1]{\textcolor{blue}{\emph{#1}}}
\newcommand\GenClass[2]{\cid{#1}\texttt{<}\cid{#2}\texttt{>}}

\newcommand\Colored[3]{\alt<#1>{\textcolor{#2}{#3}}{#3}}

\newcommand\nt[1]{\ensuremath{\langle#1\rangle}}

%%% Local Variables: 
%%% mode: latex
%%% TeX-master: nil
%%% End: 

%%% frontmatter
%% -*- coding: utf-8 -*-

\title{Functional Programming}
\subtitle{Introduction}

\author[Peter Thiemann]{Prof. Dr. Peter Thiemann}
\institute[Univ. Freiburg]{Albert-Ludwigs-Universität Freiburg, Germany}
\date{WS 2022/23}


\subtitle{Types}
\usepackage{tikz}


\begin{document}

\begin{frame}
  \titlepage
\end{frame}
\begin{frame}
  \frametitle{Contents}
  \tableofcontents{}
\end{frame}
%----------------------------------------------------------------------
\section{Predefined Types}
\begin{frame}[fragile]
  \frametitle{Predefined Types}
  \begin{itemize}
  \item \texttt{Bool}: \lstinline{True :: Bool}, \lstinline{False :: Bool}
  \item  \texttt{Char}: \lstinline{'x' :: Char}, \lstinline{'?' :: Char}, \dots
  \item \texttt{Double}, \texttt{Float}: \lstinline{3.14 :: Double}
  \item   \texttt{Integer}: \lstinline{4711 :: Integer}
  \item  \texttt{Int} --- machine integers ($\ge$ 30 bits signed
    integer)
  \item  \texttt{()}
    --- the unit type, single value \lstinline{() :: ()}
  \item  function types
  \item tuples and lists
  \item \texttt{String}:  \lstinline{"xyz" :: String}, \dots
  \item \dots
  \end{itemize}
\end{frame}
%----------------------------------------------------------------------
\section{Tuples}
\begin{frame}[fragile]
  \frametitle{Tuples}
  \begin{block}{}
\begin{lstlisting}
-- example tuples
examplePair :: (Double, Bool)  -- Double x Bool
examplePair = (3.14, False)

exampleTriple :: (Bool, Int, String) -- Bool x Int x String
exampleTriple = (False, 42, "Answer")

exampleFunction :: (Bool, Int, String) -> Bool
exampleFunction (b, i, s) = not b && length s < i
\end{lstlisting}
  \end{block}
  \begin{alertblock}{Summary}
    \begin{itemize}
    \item Syntax for tuple type like syntax for tuple values
    \item Tuples are \textbf{immutable}: in fact, \textbf{all values
        are}!\\
      Once a value is defined it cannot change! 
    \end{itemize}
  \end{alertblock}
\end{frame}
\begin{frame}
  \frametitle{Typing for Tuples}
  \begin{block}{Typing Rule}
    \begin{mathpar}
      \inferrule[Tuple]{e_1 :: t_1 \\ e_2 :: t_2 \\\dots \\ e_n :: t_n}{
        (e_1, \dots, e_n) :: (t_1, \dots, t_n)}
    \end{mathpar}
    \begin{itemize}
    \item $e_1, \dots, e_n$ are Haskell expressions
    \item $t_1, \dots, t_n$ are their respective types
    \item Then the tuple expression $(e_1, \dots, e_n)$ has the tuple
      type $(t_1, \dots, t_n)$.
    \end{itemize}
  \end{block}
\end{frame}
%----------------------------------------------------------------------
\section{Lists}
\begin{frame}
  \frametitle{Lists}
  \begin{itemize}
  \item The “duct tape” of functional programming
  \item Collections of things of the same type 
  \item 
    For any type \lstinline{x}, \lstinline{[x]} is the type of lists of \lstinline{x}s
    \\ e.g. \lstinline{[Bool]} is the type of lists of \lstinline{Bool}
  \item Syntax for list type like syntax for list values
  \item Lists are \textbf{immutable}: once a list value is defined it cannot change!
  \end{itemize}
\end{frame}
%----------------------------------------------------------------------
\begin{frame}[fragile]
  \frametitle{Constructing lists}
  \begin{block}<+->{The values of type [a] are \dots}
    \begin{itemize}
    \item either \lstinline{[]}, the empty list
    \item or \lstinline{x:xs} where \lstinline{x} has type \lstinline{a} and
      \lstinline{xs} has type \lstinline{[a]} \\
      ``\lstinline{:}'' is pronounced ``cons''
    \item \lstinline{[]} and \lstinline{(:)} are the \textbf{list constructors}
    \end{itemize}
  \end{block}
  \begin{block}<+->{Typing Rules for Lists}
    \begin{mathpar}
      \inferrule[Nil]{}{ [] :: [t] }

      \inferrule[Cons]{e_1 :: t \\ e_2 :: [t]}{(e_1 : e_2) :: [t]}
    \end{mathpar}
    \begin{itemize}
    \item The empty list can serve as a list of any type $t$
    \item If there is some $t$ such that $e_1$ has type $t$ and $e_2$
      has type $[t]$, then $(e_1:e_2)$ has type $[t]$.
    \end{itemize}
  \end{block}
\end{frame}
%----------------------------------------------------------------------
\begin{frame}[fragile]
  \frametitle{Typing Lists}
\begin{alertblock}<+->{Quiz}
    Which of the following expressions have type \lstinline{[Bool]}?
\begin{lstlisting}
  []
  True : [ ]
  True:False
  False:(False:[ ])
  (False:False):[ ]
  (False:[]):[ ]
  (True : (False : (True : []))) : (False:[]):[ ]
\end{lstlisting}
  \end{alertblock}
\end{frame}
%----------------------------------------------------------------------
\begin{frame}
  \frametitle{List shorthands}
  \begin{block}{Equivalent ways of writing a list}
    \begin{flushleft}
      \begin{tabular}{l@{\qquad---\qquad}l}
        \lstinline{1:(2:(3:[ ]))}& standard, fully parenthesized\\
        \lstinline{1:2:3:[ ]} & \lstinline{(:)} associates to the right\\
        \lstinline{[1,2,3]} &  bracketed notation
      \end{tabular}
    \end{flushleft}
  \end{block}
\end{frame}
%----------------------------------------------------------------------
\section{Pattern Matching on Lists}
\begin{frame}[fragile]
  \frametitle{Functions on lists}
  \begin{block}<+->{Definition by \textbf{pattern matching}}
\begin{lstlisting}
-- function over lists, examples for list patterns
summerize :: [String] -> String
summerize []  = "None"
summerize [x] = "Only " ++ x
summerize [x,y] = "Two things: " ++ x ++ " and " ++ y
summerize [_,_,_] = "Three things: ???"
summerize _   = "Several things."   -- wild card pattern
\end{lstlisting}
  \end{block}
  \begin{alertblock}<+->{Explanations --- patterns}
    \footnotesize{}
    \begin{itemize}
    \item patterns contain constructors and variables
    \item patterns are checked in sequence
    \item constructors are checked against argument value
    \item variables are bound to the values in
      corresponding position in the argument
    \item each variable may occur at most once in a pattern
    \item wild card pattern \verb!_! matches everything, no binding, may occur multiple times
    \end{itemize}
  \end{alertblock}

  \end{frame}
\begin{frame}[fragile]
  \frametitle{Pattern matching on lists}
\begin{alertblock}<+->{Explanations --- expressions}
    \begin{itemize}
    \item  \lstinline{(++)} \textbf{list concatenation}
    \item  \lstinline{(++)} associates to right
      % because it's more efficient
      % \begin{itemize}
      % \item \lstinline{ [1,2,3,4,5] ++ ([6,7,8,9] ++ [])} --- 10 copy
      %   operations
      % \item 
      %   \lstinline{([1,2,3,4,5] ++ [6,7,8,9]) ++ []} --- 14 copy operations,
      %   because \lstinline{[1,2,3,4,5]} is copied twice
      % \end{itemize}
    \end{itemize}
  \end{alertblock}
  \end{frame}
%----------------------------------------------------------------------
\section{Primitive Recursion, Map, and Filter}
\begin{frame}[fragile]
  \frametitle{Primitive recursion on lists}
  \begin{block}<+->{Common example: double every element in a list of numbers}
\begin{lstlisting}
-- doubles [3,6,12] = [6,12,24]
doubles :: [Integer] -> [Integer]
doubles []     = undefined
doubles (x:xs) = undefined
\end{lstlisting}
  \end{block}
  \begin{alertblock}<+->{BUT}
    Would not write it in this way --- it's a common pattern that we'll define in a library function 
  \end{alertblock}
  \begin{alertblock}<+->{}
    \begin{itemize}
    \item \lstinline{undefined} is a value of any type
    \item evaluating it yields a run-time error
    \end{itemize}
  \end{alertblock}
\end{frame}

%----------------------------------------------------------------------
\begin{frame}[fragile]
  \frametitle{map: Apply Function to Every Element of a List}
  \begin{block}<+->{Definition}
\begin{lstlisting}
-- map f [x1, x2, ..., xn] = [f x1, f x2, ..., fn]
map :: (a -> b) -> [a] -> [b]
map f []     = undefined
map f (x:xs) = undefined
\end{lstlisting}
    (map is in the standard Prelude - no need to define it)
  \end{block}
  \begin{alertblock}<+->{Define doubles in terms of map}
  \end{alertblock}
  \begin{block}<+->{}
\begin{lstlisting}
doubles xs = map double xs

double :: Integer -> Integer
double x = undefined
\end{lstlisting}
  \end{block}
\end{frame}
%----------------------------------------------------------------------
\begin{frame}[fragile]
  \frametitle{The function filter}
  Produce a list by removing all elements 
  which do not have a certain property from 
  a given list: 

\begin{lstlisting}
filter odd [1,2,3,4,5] == [1,3,5]
\end{lstlisting}

  \begin{block}{Definition}
\begin{lstlisting}
filter :: (a -> Bool) -> [a] -> [a]
filter p []       = undefined
filter p (x:xs) = undefined
\end{lstlisting}
(filter is in the standard Prelude - no need to define it)
  \end{block}
\end{frame}
%----------------------------------------------------------------------

\begin{frame}
  \frametitle{Questions?}
  \begin{center}
    \tikz{\node[scale=15] at (0,0){?};}
  \end{center}
\end{frame}


\end{document}

%%% Local Variables: 
%%% mode: latex
%%% TeX-master: t
%%% End: 
