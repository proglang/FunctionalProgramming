%% -*- coding: utf-8 -*-
\documentclass{beamer}

%% -*- coding: utf-8 -*-
\usetheme{Boadilla} % default
\useoutertheme{infolines}
\setbeamertemplate{navigation symbols}{} 

\usepackage{etex}
\usepackage{alltt}
\usepackage{pifont}
\usepackage{color}
\usepackage[utf8]{inputenc}
\usepackage{german}
\usepackage{listings}
\usepackage{hyperref}
\hypersetup{colorlinks=true}
\usepackage[final]{pdfpages}
\usepackage{url}
\usepackage{arydshln} % dashed lines

\newcommand\cmark{\ding{51}}
\newcommand\xmark{\ding{55}}

\newcommand{\nat}{\mathbf{N}}

\usepackage[all]{xy}

%% new arrow tip for xy
\newdir{|>}{!/4.5pt/@{|}*:(1,-.2)@^{>}*:(1,+.2)@_{>}}

\newcommand\cid[1]{\textup{\textbf{#1}}} % class names
\newcommand\kw[1]{\textup{\textbf{#1}}}  % key words
\newcommand\tid[1]{\textup{\textsf{#1}}} % type names
\newcommand\vid[1]{\textup{\texttt{#1}}} % value names
\newcommand\Mid[1]{\textup{\texttt{#1}}} % method names

\newcommand\TODO[1][]{{\color{red}{\textbf{TODO: #1}}}}

\newcommand\String[1]{\texttt{\dq{}#1\dq{}}}

\newcommand\ClassHead[1]{%
  \ensuremath{\begin{array}{|l|}
      \hline
      \cid{#1}
      \\\hline
    \end{array}}}
\newcommand\AbstractClass[2]{%
  \ensuremath{\begin{array}{|l|}
      \hline
      \cid{\textit{#1}}
      \\\hline
      #2
      \hline
    \end{array}}}
\newcommand\Class[2]{%
  \ensuremath{\begin{array}{|l|}
      \hline
      \cid{#1}
      \\\hline
      #2
      \hline
    \end{array}}}
\newcommand\Attribute[3][black]{\textcolor{#1}{\Param{#2}{#3}}\\}
\newcommand\Methods{\hline}
\newcommand\MethodSig[3]{\Mid{#2} (#3): \,\tid{#1}\\}
\newcommand\CtorSig[2]{\Mid{#1} (#2)\\}
\newcommand\AbstractMethodSig[3]{\Mid{\textit{#2}} (#3): \,\tid{#1}\\}
\newcommand\Param[2]{\vid{#2}:~\tid{#1}}

\lstset{%
  frame=single,
  xleftmargin=2pt,
  stepnumber=1,
  numbers=left,
  numbersep=5pt,
  numberstyle=\ttfamily\tiny\color[gray]{0.3},
  belowcaptionskip=\bigskipamount,
  captionpos=b,
  escapeinside={*'}{'*},
  language=java,
  tabsize=2,
  emphstyle={\bf},
  commentstyle=\mdseries\it,
  stringstyle=\mdseries\rmfamily,
  showspaces=false,
  showtabs=false,
  keywordstyle=\bfseries,
  columns=fullflexible,
  basicstyle=\footnotesize\CodeFont,
  showstringspaces=false,
  morecomment=[l]\%,
  rangeprefix=////,
  includerangemarker=false,
}

\newcommand\CodeFont{\sffamily}

\definecolor{lightred}{rgb}{0.8,0,0}
\definecolor{darkgreen}{rgb}{0,0.5,0}
\definecolor{darkblue}{rgb}{0,0,0.5}

\newcommand\highlight[1]{\textcolor{blue}{\emph{#1}}}
\newcommand\GenClass[2]{\cid{#1}\texttt{<}\cid{#2}\texttt{>}}

\newcommand\Colored[3]{\alt<#1>{\textcolor{#2}{#3}}{#3}}

\newcommand\nt[1]{\ensuremath{\langle#1\rangle}}

%%% Local Variables: 
%%% mode: latex
%%% TeX-master: nil
%%% End: 

%%% frontmatter
%% -*- coding: utf-8 -*-

\title{Functional Programming}
\subtitle{Introduction}

\author[Peter Thiemann]{Prof. Dr. Peter Thiemann}
\institute[Univ. Freiburg]{Albert-Ludwigs-Universität Freiburg, Germany}
\date{WS 2022/23}


\subtitle{Type definitions}
\usepackage{tikz}


\begin{document}

\begin{frame}
  \titlepage
\end{frame}
%----------------------------------------------------------------------
\begin{frame}[fragile]
  \frametitle{Type aliases}
Explaining the meaning of data in comments is bad!

Introduce new, self explaining types.
\begin{lstlisting}
type Name = String
type Title = String
type Year = Int
type Age = Int

type User = (Name,    Year)
--           name,   year of birth 
type Film = (Title, Age)
--                  ^ fsk           
type Purchase = (Name    -- use name
                , Title  -- item name
                , Year)  -- date of purchase
users :: [User]
\end{lstlisting}
\end{frame}
%----------------------------------------------------------------------
\begin{frame}[fragile]
  \frametitle{Datatypes}
  \begin{block}{Example scenario}
    \begin{itemize}
    \item model a card game (hearts)
    \item represent the game items!
    \item define game logic on the representations!
    \end{itemize}
  \end{block}
\end{frame}
%----------------------------------------------------------------------
\begin{frame}
  \frametitle{Intermezzo: The game}
  \begin{block}<+->{Microsoft Hearts (\href{https://en.wikipedia.org/wiki/Microsoft_Hearts}{Wikipedia link})}
    \begin{itemize}
    \item computer game based on card game ``Hearts''
    \item included in Windows~3.1 through Windows~7
    \item discontinued
    \end{itemize}    
  \end{block}
\end{frame}
\begin{frame}
  \frametitle{Gameplay}
  \begin{block}<+->{}
    \begin{itemize}
    \item four players (three simulated)
    \item trick-taking game
    \item each player plays one card to a trick
    \item trick won by highest card of the suit led; no Trump!
    \item suit must be followed
    \item Heart cannot lead until
      \begin{itemize}
      \item either Heart has been broken --- a player played Heart
      \item or the leading player has only Heart
      \end{itemize}
    \item points are scored by any Hearts (1 point) and the Queen of Spades (13 points) 
    \end{itemize}
  \end{block}
  \begin{block}<+->{Objective}
    \begin{itemize}
    \item Avoid gaining points \textbf{or} gain all 26 points
    \end{itemize}
  \end{block}
  
\end{frame}
%----------------------------------------------------------------------
\begin{frame}
  \frametitle{Data model for card games}
  \begin{itemize}
  \item A card has a \textbf{Suit} and a \textbf{Rank}
  \item A card beats another card if it has the same suit, but higher rank
  \item Todo:
    \begin{itemize}
    \item represent cards
    \item define when one card beats another
    \item define a function that chooses a beating card from a hand of
      cards, if possible
    \end{itemize}
  \end{itemize}
\end{frame}
\begin{frame}[fragile]
  \frametitle{Model using algebraic datatypes}
  \begin{block}{A card has a Suit}
\begin{lstlisting}
data Suit = Spades | Hearts | Diamonds | Clubs
\end{lstlisting}
  \end{block}
  \begin{alertblock}{Explanation}
    \begin{itemize}
    \item define an \emph{algebraic datatype}
    \item new type consisting of (exactly) four values
    \item \lstinline{Suit}: the name of the new type
    \item \lstinline{Spades}, \lstinline{Hearts}, \dots: the names of its
      \textbf{constructors}
    \item constructors can be used in expressions and patterns
    \item names of types and constructors must be capitalized
    \end{itemize}
  \end{alertblock}
\end{frame}
%----------------------------------------------------------------------
\begin{frame}[fragile]
  \frametitle{Printing algebraic datatypes}
\begin{verbatim}
Main> Spades

<interactive>:3:1:
  No instance for (Show Suit) arising from
  a use of `print'
  Possible fix: [...] 
\end{verbatim}

  \begin{alertblock}{Oops!}
    \begin{itemize}
    \item Haskell does not know how to print a \texttt{Suit}
    \item but we can ask for a default (or write our own printer)
    \end{itemize}
  \end{alertblock}
\end{frame}

\begin{frame}[fragile,fragile]
  \frametitle{Printing derived}
\begin{lstlisting}
data Suit = Spades | Hearts | Diamonds | Clubs
  deriving (Show) -- makes `Suit' printable
\end{lstlisting}
Defines a function \lstinline{show} for \lstinline{Suit}, which is
automatically called by Haskell's printer
\begin{verbatim}
Main> Spades
Spades
Main> show Spades
"Spades"
Main> :t show 
show :: Show a => a -> String
\end{verbatim}
\begin{exampleblock}{Remark}
  \begin{itemize}
  \item \lstinline{Show} is a \emph{type class}
  \item a type class associates one or more functions with a type; in
    case of \lstinline{Show}, the function is \lstinline{show}
  \end{itemize}
\end{exampleblock}
\end{frame}
\begin{frame}[fragile,fragile]
  \frametitle{Functions on data types}
   Each suit has a color:
\begin{lstlisting}
 data Color = Black | Red
  deriving (Show)
\end{lstlisting}
   Define a color function by pattern matching
\begin{lstlisting}
color :: Suit -> Color
color = undefined
\end{lstlisting}
\end{frame}
\begin{frame}[fragile,fragile]
  \frametitle{More data}
  A card has a suit and a \textbf{rank}:
\begin{lstlisting}
data Rank = Numeric Integer | Jack | Queen | King | Ace
  deriving Show
\end{lstlisting}
The constructor \lstinline{Numeric} is different: it takes an argument.
\begin{verbatim}
Main> :t Numeric
Numeric :: Integer -> Rank
\end{verbatim}
\end{frame}
\begin{frame}[fragile]
  \frametitle{Comparing ranks}
  \begin{block}<+->{Situation}
    \begin{itemize}
    \item Let \texttt{r2} be the highest rank on the table
    \item Let \texttt{r1} be the card played
    \item Assuming the suits match, does \texttt{r1} get the trick?
    \end{itemize}
  \end{block}
  \begin{block}<+->{Need an ordering of ranks}
\begin{lstlisting}
-- |rankBeats r1 r2
-- returns True, if r1 beats r2
-- i.e. r1 is strictly greater than r2
rankBeats :: Rank -> Rank -> Bool
rankBeats r1 r2 = undefined
\end{lstlisting}
  \end{block}
\end{frame}
\begin{frame}[fragile]
  \frametitle{Ordering ranks by pattern matching}
\begin{lstlisting}
-- rankBeats r1 r2 returns True, if r1 beats r2
rankBeats :: Rank -> Rank -> Bool
rankBeats _  Ace    = False
rankBeats Ace _     = True
rankBeats _ King    = False
rankBeats King _    = True
rankBeats _  Queen  = False
rankBeats Queen _   = True
rankBeats _ Jack    = False
rankBeats Jack _    = True
rankBeats (Numeric n1) (Numeric n2) = n1 > n2
-- pattern match on Numeric constructor yields its argument
\end{lstlisting}
\end{frame}

\begin{frame}[fragile]
  \frametitle{Letting Haskell order the ranks}
  \begin{itemize}
  \item definition of \texttt{rankBeats} is repetitive
  \item boilerplate code
  \item let Haskell generate it for us!
  \end{itemize}
  \begin{block}<+->{Deriving an order}
    The comparison operators \lstinline{<=}, \lstinline{<} etc are overloaded and can be extended to
    new types
\begin{lstlisting}
data Rank = Numeric Integer | Jack | Queen | King | Ace
  deriving (Show, Ord)

rankBeats' r1 r2 = r1 > r2
\end{lstlisting}
  \end{block}
  \begin{exampleblock}{Remark}
    \begin{itemize}
    \item \lstinline{Ord} is another type class that governs
      \lstinline{<}, \lstinline{<=}, etc
    \end{itemize}
  \end{exampleblock}
\end{frame}

\begin{frame}[fragile]
  \frametitle{Oops\dots}
\begin{verbatim}
M04Cards.hs:17:27: error:
    • No instance for (Eq Rank)
        arising from the 'deriving' clause of a data type declaration
      Possible fix:
        use a standalone 'deriving instance' declaration,
          so you can specify the instance context yourself
    • When deriving the instance for (Ord Rank)
\end{verbatim}
  \begin{block}<2->{Explanation}
    Type class \lstinline{Ord} defines \texttt{<} and then
\begin{lstlisting}
x <= y = x < y || x == y
\end{lstlisting}
    but how do we compare two ranks for equality?
  \end{block}
\end{frame}

\begin{frame}[fragile]
  \frametitle{Equality of ranks}
  \begin{itemize}
  \item could be defined by pattern matching, but
  \item let Haskell generate this boilerplate code for us!
  \end{itemize}
  \begin{block}<+->{Deriving equality}
    \lstinline{==}, \lstinline{/=} are overloaded and can be extended to
    new types
\begin{lstlisting}
data Rank = Numeric Integer | Jack | Queen | King | Ace
  deriving (Show, Eq, Ord)

rankBeats' r1 r2 = r1 > r2
\end{lstlisting}
  \end{block}
  \begin{block}<+->{Are they the same?}
    How do we know that \lstinline{rankBeats} $=$ \lstinline{rankBeats'}?
    Let's defer that.

  \end{block}
\end{frame}

\begin{frame}[fragile]
  \frametitle{Cards, finally}
  A card has a \lstinline{Suit} and a \lstinline{Rank}
\begin{lstlisting}
data Card = Card Rank Suit
  deriving (Show)

rank :: Card -> Rank
rank (Card r s) = r

suit :: Card -> Suit
suit (Card r s) = s
\end{lstlisting}
  \begin{itemize}
  \item \lstinline{Card} has single constructor with two parameters
  \item  (in principle, a tuple with a special name)
  \item \lstinline{rank}, \lstinline{suit} are \textbf{selector functions}
  \end{itemize}
\end{frame}

\begin{frame}[fragile]
  \frametitle{Alternative definition of Cards}
  There is a way to define the type along with its selector
  functions using Haskell's (hated) records types:
\begin{lstlisting}
data Card = Card { rank :: Rank, suit :: Suit }
  deriving (Show)
\end{lstlisting}
  \begin{itemize}
  \item defines type \lstinline{Card} and its constructor
    \lstinline{Card}
  \item defines selectors \lstinline{rank :: Card -> Rank} and
    \lstinline{suit :: Card -> Suit}
  \item we can use \emph{record notation} to construct values:
\begin{lstlisting}[numbers=none]
queenOfSpades = Card{ rank= Queen, suit= Spades }
\end{lstlisting}
  \item and \emph{record updates}:
\begin{lstlisting}[numbers=none]
queenOfHearts = queenOfSpades { suit= Hearts }
\end{lstlisting}
  \end{itemize}
\end{frame}
\begin{frame}[fragile]
  \frametitle{Comparing Cards}
  A card \textbf{beats} another card, if it has the same
  suit, but a higher rank
\begin{lstlisting}
cardBeats :: Card -> Card -> Bool
cardBeats givenCard c = suit givenCard == suit c
                        && rankBeats (rank givenCard) (rank c)
\end{lstlisting}
\end{frame}
\begin{frame}[fragile]
  \frametitle{Hand of Cards}
\begin{lstlisting}
type Hand = [Card]

chooseCard :: Card -> Hand -> Card
chooseCard givenCard h = undefined
\end{lstlisting}
  To develop \texttt{chooseCard} refine \texttt{h} by pattern matching
\end{frame}
\begin{frame}[fragile]
  \frametitle{Choose a card}
\begin{lstlisting}
type Hand = [Card]

chooseCard :: Card -> Hand -> Card
chooseCard givenCard [] = undefined -- ???
chooseCard givenCard (x:xs) = undefined
\end{lstlisting}
  \begin{itemize}
  \item What should we do if the hand is empty?
  \item Avoid by defining only non-empty hands!
  \end{itemize}
\end{frame}
\begin{frame}[fragile]
  \frametitle{Non-empty hands}
\begin{lstlisting}
data Hand = Last Card | Next Card Hand
  deriving (Show, Eq)
\end{lstlisting}
  \begin{itemize}
  \item  Recursive datatype definition
  \item  \lstinline{Last Card} is the \emph{base case}
  \end{itemize}
\end{frame}
\begin{frame}[fragile]
  \frametitle{Get card from non-empty hand}
  \begin{itemize}
  \item  A \lstinline{Hand} is never empty
  \item  Thus we can always obtain a card
  \end{itemize}
\begin{lstlisting}
topCard :: Hand -> Card
topCard (Last c)  = c
topCard (Next c _) = c
\end{lstlisting}
\end{frame}
\begin{frame}[fragile]
  \frametitle{Choosing from non-empty hand}
\begin{lstlisting}
-- choose a beating card, if possible
chooseCard :: Card -> Hand -> Card
chooseCard = undefined
\end{lstlisting}
\end{frame}
\begin{frame}[fragile]
  \frametitle{Choosing from non-empty hand}
\begin{lstlisting}
-- choose a beating card, if possible
chooseCard :: Card -> Hand -> Card
chooseCard gc (Last c) = c -- may beat, or not
chooseCard gc (Next c h) | cardBeats gc c = c
                         | otherwise      = chooseCard gc h
\end{lstlisting}
\end{frame}
%----------------------------------------------------------------------

\begin{frame}
  \frametitle{Questions?}
  \begin{center}
    \tikz{\node[scale=15] at (0,0){?};}
  \end{center}
\end{frame}


\end{document}

%%% Local Variables: 
%%% mode: latex
%%% TeX-master: t
%%% End: 
