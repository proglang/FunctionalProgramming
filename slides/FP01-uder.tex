%% -*- coding: utf-8 -*-
\documentclass{beamer}

%% -*- coding: utf-8 -*-
\usetheme{Boadilla} % default
\useoutertheme{infolines}
\setbeamertemplate{navigation symbols}{} 

\usepackage{etex}
\usepackage{alltt}
\usepackage{pifont}
\usepackage{color}
\usepackage[utf8]{inputenc}
\usepackage{german}
\usepackage{listings}
\usepackage{hyperref}
\hypersetup{colorlinks=true}
\usepackage[final]{pdfpages}
\usepackage{url}
\usepackage{arydshln} % dashed lines

\newcommand\cmark{\ding{51}}
\newcommand\xmark{\ding{55}}

\newcommand{\nat}{\mathbf{N}}

\usepackage[all]{xy}

%% new arrow tip for xy
\newdir{|>}{!/4.5pt/@{|}*:(1,-.2)@^{>}*:(1,+.2)@_{>}}

\newcommand\cid[1]{\textup{\textbf{#1}}} % class names
\newcommand\kw[1]{\textup{\textbf{#1}}}  % key words
\newcommand\tid[1]{\textup{\textsf{#1}}} % type names
\newcommand\vid[1]{\textup{\texttt{#1}}} % value names
\newcommand\Mid[1]{\textup{\texttt{#1}}} % method names

\newcommand\TODO[1][]{{\color{red}{\textbf{TODO: #1}}}}

\newcommand\String[1]{\texttt{\dq{}#1\dq{}}}

\newcommand\ClassHead[1]{%
  \ensuremath{\begin{array}{|l|}
      \hline
      \cid{#1}
      \\\hline
    \end{array}}}
\newcommand\AbstractClass[2]{%
  \ensuremath{\begin{array}{|l|}
      \hline
      \cid{\textit{#1}}
      \\\hline
      #2
      \hline
    \end{array}}}
\newcommand\Class[2]{%
  \ensuremath{\begin{array}{|l|}
      \hline
      \cid{#1}
      \\\hline
      #2
      \hline
    \end{array}}}
\newcommand\Attribute[3][black]{\textcolor{#1}{\Param{#2}{#3}}\\}
\newcommand\Methods{\hline}
\newcommand\MethodSig[3]{\Mid{#2} (#3): \,\tid{#1}\\}
\newcommand\CtorSig[2]{\Mid{#1} (#2)\\}
\newcommand\AbstractMethodSig[3]{\Mid{\textit{#2}} (#3): \,\tid{#1}\\}
\newcommand\Param[2]{\vid{#2}:~\tid{#1}}

\lstset{%
  frame=single,
  xleftmargin=2pt,
  stepnumber=1,
  numbers=left,
  numbersep=5pt,
  numberstyle=\ttfamily\tiny\color[gray]{0.3},
  belowcaptionskip=\bigskipamount,
  captionpos=b,
  escapeinside={*'}{'*},
  language=java,
  tabsize=2,
  emphstyle={\bf},
  commentstyle=\mdseries\it,
  stringstyle=\mdseries\rmfamily,
  showspaces=false,
  showtabs=false,
  keywordstyle=\bfseries,
  columns=fullflexible,
  basicstyle=\footnotesize\CodeFont,
  showstringspaces=false,
  morecomment=[l]\%,
  rangeprefix=////,
  includerangemarker=false,
}

\newcommand\CodeFont{\sffamily}

\definecolor{lightred}{rgb}{0.8,0,0}
\definecolor{darkgreen}{rgb}{0,0.5,0}
\definecolor{darkblue}{rgb}{0,0,0.5}

\newcommand\highlight[1]{\textcolor{blue}{\emph{#1}}}
\newcommand\GenClass[2]{\cid{#1}\texttt{<}\cid{#2}\texttt{>}}

\newcommand\Colored[3]{\alt<#1>{\textcolor{#2}{#3}}{#3}}

\newcommand\nt[1]{\ensuremath{\langle#1\rangle}}

%%% Local Variables: 
%%% mode: latex
%%% TeX-master: nil
%%% End: 

%%% frontmatter
%% -*- coding: utf-8 -*-

\title{Functional Programming}
\subtitle{Introduction}

\author[Peter Thiemann]{Prof. Dr. Peter Thiemann}
\institute[Univ. Freiburg]{Albert-Ludwigs-Universität Freiburg, Germany}
\date{WS 2022/23}


\usepackage{tikz}

\subtitle{Part I}
\date{Uder, 30.05.2019}

\begin{document}

\begin{frame}
  \titlepage
\end{frame}


%----------------------------------------------------------------------

\begin{frame}
  \frametitle{Contents}
  \begin{itemize}
  \item Basics of functional programming using Haskell
  \item Haskell development tools
  \item Writing Haskell programs
  \item Using Haskell libraries
  \item Your first Haskell project
  \end{itemize}
\end{frame}


%----------------------------------------------------------------------
\section{Introduction}
\begin{frame}
  \frametitle{What is Functional Programming?}
  \begin{block}<+->{A different approach to programming}
    \begin{LARGE}
      \begin{center}\bf
        ~
        \\[1ex]
        Functions and values
        \\[2ex]
        {\normalsize rather than}
        \\[2ex]
        Assignments and addresses
        \\[1ex]
        ~
      \end{center}
    \end{LARGE}
  \end{block}
  \begin{alertblock}<+->{\bf It will make you a better programmer}
    ~
  \end{alertblock}
\end{frame} 
%----------------------------------------------------------------------

\begin{frame}[fragile]
  \frametitle{Functional vs Imperative Programming: Variables}
  \begin{block}<+->{Functional (Haskell)}
    \begin{minipage}[t]{0.45\linewidth}
\begin{lstlisting}[language=Haskell]
x :: Int
x = 5
\end{lstlisting}
    \end{minipage}
    ~
    \begin{minipage}[t]{0.45\linewidth}
      Variable \texttt{x} has value \texttt{5} forever
    \end{minipage}
  \end{block}
  \begin{block}<+->{Imperative (Java / C)}
    \begin{minipage}[t]{0.45\linewidth}
\begin{lstlisting}[language=Java]
int x = 5;
...
x = x+1;
\end{lstlisting}
    \end{minipage}
    ~
    \begin{minipage}[t]{0.45\linewidth}
      Variable \texttt{x} can change its content over time
    \end{minipage}
  \end{block}
\end{frame}

\begin{frame}[fragile]
  \frametitle{Functional vs Imperative Programming: Functions}
  \begin{block}<+->{Functional (Haskell)}
    \begin{minipage}[t]{0.45\linewidth}
\begin{lstlisting}[language=Haskell]
f :: Int -> Int -> Int
f x y = 2*x + y

f 42 16 // always 100
\end{lstlisting}
    \end{minipage}
    ~
    \begin{minipage}[t]{0.45\linewidth}
      Return value of a function \textbf{only} depends on its inputs
    \end{minipage}
  \end{block}
  \begin{block}<+->{Imperative (Java)}
    \begin{minipage}[t]{0.45\linewidth}
\begin{lstlisting}[language=Java]
boolean flag;
static int f (int x, int y) {
  return flag ? 2*x + y , 2*x - y;
}

int z = f (42, 16); // who knows?
\end{lstlisting}
    \end{minipage}
    ~
    \begin{minipage}[t]{0.45\linewidth}
      Return value depends on non-local variable \texttt{flag}
    \end{minipage}
  \end{block}
\end{frame}

\begin{frame}[fragile]
  \frametitle{Functional vs Imperative Programming: Laziness}
  \begin{block}<+->{Haskell}
    \begin{minipage}[t]{0.45\linewidth}
\begin{lstlisting}[language=Haskell]
x = expensiveComputation
g anotherExpensiveComputation
\end{lstlisting}
    \end{minipage}
    ~
    \begin{minipage}[t]{0.45\linewidth}
      \begin{itemize}
      \item The expensive computation will only happen if \texttt{x}
        is ever used.
      \item Another expensive computation will only happen if
        \texttt{g} uses its argument.
      \end{itemize}
    \end{minipage}
  \end{block}
  \begin{block}<+->{Java}
    \begin{minipage}[t]{0.45\linewidth}
\begin{lstlisting}[language=Java]
int x = expensiveComputation;
g (anotherExpensiveComputation)
\end{lstlisting}
    \end{minipage}
    ~
    \begin{minipage}[t]{0.45\linewidth}
      \begin{itemize}
      \item Both expensive computations will happen anyway.
      \item Laziness can be simulated, but it's complex!
      \end{itemize}
    \end{minipage}
  \end{block}
\end{frame}

%----------------------------------------------------------------------
\begin{frame}
  \frametitle{Many features that make programs more concise}
  \begin{itemize}
  \item Pattern Matching
  \item Higher-order functions
  \item Algebraic datatypes
  \item Polymorphic types
  \item Parametric overloading
  \item Type inference
  \item Monads \& friends (for IO, concurrency, \dots)
  \item Comprehensions
  \item Metaprogramming
  \item Domain specific languages
  \item \dots
  \end{itemize}
\end{frame}
%----------------------------------------------------------------------
\section{Types}
\begin{frame}[fragile]
  \frametitle{Predefined Types}
  \framesubtitle{Every Haskell value has a type}
  \begin{center}
    \begin{tabular}{l@{ --- }l}
      \texttt{Bool}
      & \lstinline|True :: Bool|, \lstinline|False :: Bool| \\
      \texttt{Char}
      &\lstinline|'x' :: Char|, \lstinline|'?' :: Char|, \dots\\
      \texttt{Double}, \texttt{Float}
      & \lstinline|3.14 :: Double| \\
      \texttt{Integer}
      & \lstinline|4711 :: Integer| \\
      \texttt{Int}
      &machine integers ($\ge$ 30 bits signed integer) \\
      \texttt{()}
      & the \alert{unit type}, single value \lstinline|() :: ()| \\
      \lstinline|a -> b|
      & function types \\
      \lstinline|(a, b)|
      &tuple types \\
      \lstinline|[a]|
      &list types \\
      \lstinline|String|
      &  \lstinline|"xyz" :: String|, \dots
    \end{tabular}
  \end{center}
\end{frame}
% ----------------------------------------------------------------------
\begin{frame}[fragile]
  \frametitle{Functions}
\begin{block}{Examples.hs}
\begin{lstlisting}
dollarRate = 1.3671

-- |convert EUR to USD
usd euros = euros * dollarRate
\end{lstlisting}
  \end{block}
  \begin{itemize}
  \item \lstinline{dollarRate} defines a constant
  \item \lstinline{usd} is a function 
  \item Its type \lstinline{Double -> Double} is \alert{inferred} by
    the Haskell compiler
  \item To compute, a function call \lstinline{usd arg} is replaced by
    the right hand side of its definition\\
    \lstinline{usd arg} $\rightarrow$ \lstinline{arg * dollarRate}
    $\rightarrow$ \lstinline{arg * 1.3671} $\rightarrow$ \dots
  \end{itemize}
\end{frame}
% ----------------------------------------------------------------------
\begin{frame}[fragile]
  \frametitle{Tuples}
  \begin{block}{}
\begin{lstlisting}[language=Haskell]
-- example tuples
examplePair :: (Double, Bool)  -- Double x Bool
examplePair = (3.14, False)

exampleTriple :: (Bool, Int, String) -- Bool x Int x String
exampleTriple = (False, 42, "Answer")

exampleFunction :: (Bool, Int, String) -> Bool
exampleFunction (b, i, s) = not b && length s < i
\end{lstlisting}
  \end{block}
  \begin{alertblock}{Summary}
    \begin{itemize}
    \item Syntax for tuple type like syntax for tuple values
    \item Tuples are \textbf{immutable}: in fact, \textbf{all values
        are}!\\
      Once a value is defined it cannot change! 
    \end{itemize}
  \end{alertblock}
\end{frame}
%----------------------------------------------------------------------

\begin{frame}
  \frametitle{Typing for Tuples}
  \begin{block}{Typing Rule}
    \begin{mathpar}
      \inferrule[Tuple]{e_1 :: t_1 \\ e_2 :: t_2 \\\dots \\ e_n :: t_n}{
        (e_1, \dots, e_n) :: (t_1, \dots, t_n)}
    \end{mathpar}
    If
    \begin{itemize}
    \item $e_1, \dots, e_n$ are Haskell expressions
    \item $t_1, \dots, t_n$ are their respective types
    \item Then the tuple expression $(e_1, \dots, e_n)$ has the tuple
      type $(t_1, \dots, t_n)$.
    \end{itemize}
  \end{block}
\end{frame}
%----------------------------------------------------------------------

\begin{frame}
  \frametitle{Lists}
  \begin{itemize}
  \item The “duct tape” of functional programming
  \item Collections of things of the same type 
  \item 
    For any type \lstinline{a}, \lstinline{[a]} is the type of lists
    with elements of type \lstinline{a}
    \\ e.g. \lstinline{[Bool]} is the type of lists of \lstinline{Bool}
  \item Syntax for list type like syntax for list values
  \item Lists are \textbf{immutable}: once a list value is defined it cannot change!
  \end{itemize}
\end{frame}
%----------------------------------------------------------------------
\begin{frame}[fragile]
  \frametitle{Constructing lists}
  \begin{block}<+->{The values of type [a] are \dots}
    \begin{itemize}
    \item either \lstinline{[]}, the empty list
    \item or \lstinline{x:xs} where \lstinline{x} has type \lstinline{a} and
      \lstinline{xs} has type \lstinline{[a]} \\
      ``\lstinline{:}'' is pronounced ``cons''
    \item \lstinline{[]} and \lstinline{(:)} are the \textbf{list constructors}
    \end{itemize}
  \end{block}
  \begin{block}<+->{Typing Rules for Lists}
    \begin{mathpar}
      \inferrule[Nil]{}{ [] :: [t] }

      \inferrule[Cons]{e_1 :: t \\ e_2 :: [t]}{(e_1 : e_2) :: [t]}
    \end{mathpar}
    \begin{itemize}
    \item The empty list can serve as a list of any type $t$
    \item If there is some $t$ such that $e_1$ has type $t$ and $e_2$
      has type $[t]$, then $(e_1:e_2)$ has type $[t]$.
    \end{itemize}
  \end{block}
\end{frame}
%----------------------------------------------------------------------
\begin{frame}[fragile]
  \frametitle{Typing Lists}
\begin{alertblock}<+->{Quiz Time}
    Which of the following expressions have type \lstinline{[Bool]}?
\begin{lstlisting}[language=Haskell]
  [ ]
  True : [ ]
  True : False
  False : (False : [ ])
  (False : False) : [ ]
  (False : []) : [ ]
  (True : (False : (True : []))) : (False:[]):[ ]
\end{lstlisting}
  \end{alertblock}
\end{frame}
%----------------------------------------------------------------------
\section{Pattern Matching on Lists}
\begin{frame}[fragile]
  \frametitle{Functions on lists}
  \begin{block}<+->{Definition by \textbf{pattern matching}}
\begin{lstlisting}
-- double every element of a list of integers
-- doubles [3,6,12] = [6,12,24]
doubles :: [Integer] -> [Integer]
doubles []     = []
doubles (x:xs) = (2 * x) : doubles xs
\end{lstlisting}
  \end{block}
  \begin{alertblock}<+->{Explanations --- patterns}
    \footnotesize{}
    \begin{itemize}
    \item patterns contain constructors and variables
    \item patterns are checked in sequence
    \item constructors are checked against argument value
    \item variables are bound to the values in
      corresponding position in the argument
    \item each variable may occur at most once in a pattern
    \item wild card pattern \verb!_! matches everything, no binding, may occur multiple times
    \end{itemize}
  \end{alertblock}

  \end{frame}

%----------------------------------------------------------------------
\begin{frame}
  \frametitle{References}
  \begin{itemize}
  \item Paper by the original developers of Haskell in the conference on History of
    Programming Languages (HOPL III): \href{A History
      of Haskell: Being Lazy with
      Class}{http://dl.acm.org/citation.cfm?id=1238856}
  \item The Haskell home page: \url{http://www.haskell.org}
  \item Haskell libraries repository:
    \url{https://hackage.haskell.org/}
  \item Haskell Tool Stack: \url{https://docs.haskellstack.org/en/stable/README/}
  \end{itemize}
\end{frame}

%----------------------------------------------------------------------

\begin{frame}
  % \frametitle{Questions?}
  \begin{center}
    \tikz{\node[scale=15] at (0,0){?};}
  \end{center}
\end{frame}



\end{document}

%%% Local Variables: 
%%% mode: latex
%%% TeX-master: t
%%% End: 
